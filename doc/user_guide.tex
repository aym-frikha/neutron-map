\documentclass[12pt]{article}
\usepackage[utf8]{inputenc}
\usepackage[margin=3em]{geometry}
\usepackage{tabularx}
\usepackage{graphicx}
\usepackage{xcolor}
\usepackage{lmodern}
\usepackage{listings}
\lstset{language=,
    basicstyle=\sffamily\footnotesize,
    xleftmargin=20pt,
    xrightmargin=20pt,
    numbers=left,
    stepnumber=1, % le pas des numeros de ligne
    numbersep=10pt,
    breaklines=true,
    tabsize=4,
    %frame=single,
    showspaces=false,
    showstringspaces=false
    inputencoding=utf8, %put1 d'utf8 dans listing
    extendedchars=true, %put1 d'utf8 dans listing
    columns=fullflexible, % suppression des espaces autour des deux-points
    escapechar=`, % permet d'insérer du §code latex§
    backgroundcolor=\color{lightgray},
    keywordstyle=\hlc,
    morecomment=[l]{\#}, % commentaires shell en ligne
    commentstyle=\itshape\color{darkgray},
}



\begin{document}

\title{MAP:\\User guide}
\maketitle
\section{Installation}

To install MAP as your openstack plugin, you must ensure the following steps
are respected.\\
First of all, get the last version of this plugin and paste main folder in
your neutron's plugins folder (usually /opt/stack/neutron/neutron/plugins/).
Rename this folder 'map'.\\

If you installed devstack in /opt/stack, you can uses 'finalize.sh' script, found in 'install' folder.
If necessary, change variables at the top of the script.
\begin{verbatim}
cd install && sudo ./finalize.sh
\end{verbatim}

If you openstack installation is more specific, you must follow the sames steps
than the script described bellow.

\subsection{MAP as core plugin}
Starting MAP as neutron's main plugin:
Replace \textit{core\_plugin} parameter by:
%\begin{verbatim}
%core_plugin = neutron.plugins.openvswitch.ovs_neutron_plugin.OVSNeutronPluginV2
%\end{verbatim}
%by~:
\begin{verbatim}
core_plugin = neutron.plugins.map.dynamic_plugin.DynamicNeutronPlugin
\end{verbatim}
in neutron's configuration file (default: /etc/neutron/neutron.conf)

\subsection{MAP configuration}
MAP needs a configuration file to run. 'example-config.ini' is a configuration
file example in 'install' folder. This file can be stored in /etc/neutron. For
example~:

\begin{verbatim}
mkdir -p /etc/neutron/plugins/map/
cp example-config.ini /etc/neutron/plugins/map/map.ini
\end{verbatim}
Then adapt it to your own configuration.\\
When starting neutron, this configuration file must be called. If screen is
used (devstack), replace~:
\begin{verbatim}
cd /opt/stack/neutron && python /usr/local/bin/neutron-server --config-file \
  /etc/neutron/neutron.conf --config-file \
  /etc/neutron/plugins/openvswitch/ovs_neutron_plugin.ini
\end{verbatim}
by~:
\begin{verbatim}
cd /opt/stack/neutron && python /usr/local/bin/neutron-server --config-file \
  /etc/neutron/neutron.conf --config-file /etc/neutron/plugins/map/map.ini
\end{verbatim}
in stack-screenrc (default: /opt/stack/devstack/stack-screenrc)

\subsection{Data storage}
MAP needs to store some data, in particular on network or for drivers. Usually,
services data are stored in /opt/stack/data/\textless service\_name\textgreater / \\
MAP needs a 'network' folder and netconf driver needs a 'netconf' folder.
'install' folder contains example. As an exemple, this is how it can be done~:
\begin{verbatim}
mkdir -p /opt/stack/data/neutron/plugins/map/
mv network /opt/stack/data/neutron/plugins/map/
mv netconf /opt/stack/data/neutron/plugins/map/
\end{verbatim}
New folder can as well be used~:
\begin{verbatim}
mkdir -p /opt/stack/data/neutron/plugins/map/network
mkdir /opt/stack/data/neutron/plugins/map/netconf
\end{verbatim}
MAP's configuration file may be adapted to refer those folders.

\subsection{Dependencies}
Here is a list of map dependencies:
\begin{itemize}
    \item pyxb
    \item ncclient
\end{itemize}
Install dependencies~:
\begin{verbatim}
sudo pip install pyxb
sudo git clone git://github.com/leopoul/ncclient
(cd ncclient && patch -p1 < ../dynamips_compat.patch && python setup.py install)
\end{verbatim}
'install' folder contains a patch to apply on ncclient before installing it.
This patch is required to work with dynagen/gns3 equipments.\\
{\large\textbf{WARNING:}} \textbf{This patch will probably induce ncclient
inablity to communicate with real equipments. For our test however, we had to use
it to communicate with cisco emulated equipments. Two major points are changed
by this patch: \begin{itemize}
\item Spaces add at the end of tags are removed (\textless tag /\textgreater
=\textgreater \textless tag/\textgreater)
\item Namespace used is different ('urn:ietf:params:xml:ns:netconf:base:1.0'\\
becomes 'urn:ietf:params:netconf:base:1.0')
\end{itemize}}

\section{Administrate}

\subsection{Network modifications}
When network is modified, topology file must be updated. In 'install/network'
folder, a '.db' file is given as an example~:
\begin{verbatim}
NODE,R1,L3
NODE,SW3,L2
NODE,SW2,L2

LINK,R1,FastEthernet0/0,SW2,FastEthernet1/4
LINK,R1,FastEthernet1/0,SW1,FastEthernet1/3
\end{verbatim}
This topology file is a csv containing Nodes declarations with keyword 'NODE'
followed by nodes name and its switching level (Layer 2 or Layer 3). Links are
declared in this file as well, with 'LINK' keyword, then
origin Node name and port followed by destination Node name and port.\\
At least one driver must also get credentials to access and modify device
configuration. If you use netconf, add new devices in device file. An example
device file is also provided in 'install/network'~:
\begin{verbatim}
#name,ip,port,login,password,netconf syntaxe
R1,132.208.13.62,22,charles,behappy,gns3
\end{verbatim}
As commented above, a device is a comma separated value line. Fields are in order:
name, ip, port, login, password and a keyword that permits to identify netconf's
format used to communicate with the device. Netconf formats are stored in
syntaxe files. To see an example~:'install/netconf/syntaxe/'. Stx files actually
contain python dictionarys with python string values. 'Command' key refers to
syntaxe of a command. 'Wrapper' is used once for all commands.\\
\textbf{Currently, neutron must be restarted after any modification on those
files!}\\[1\baselineskip] Referenced files above are configured in MAP configuration file. Format used
may be changed.
\begin{itemize}
    \item A single file will probably be created for all drivers, adding at
    least a 'driver' field
    \item Topology file and driver file may be merged
    \item If variable syntaxes are found for netconf, stx files may change to
    cover various syntaxes needs.
\end{itemize}
This documentation should be updated accordingly.

\end{document}
